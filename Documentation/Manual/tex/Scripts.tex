
\documentclass[../D+Manual.tex]{subfiles}
\begin{document}

\chapter{Lua Scripts} \label{chp:scripts}

\begin{quote}
	It's nice to finally get scripts offered to me that aren't the ones Tom Hanks wipes his butt with.\\
	\hspace*{\fill} \textit{Jim Carrey}
\end{quote}


\lstset{style=Luastyle,}

Scripting in D+ is done with the \href{http://www.lua.org/}{Lua language}.
Scripts can be used to create \hyperref[chp:Symmetries]{Symmetries} (covered in chapter \ref{chp:Symmetries}) and to do repetitious work.
It can also be used as a general \href{http://www.lua.org/}{\texttt{Lua}} environment.
There are several examples available in the \path{./LuaScripts/} directory, where D+ is installed.
This chapter will address the bound D+ functions that have been exposed for use.

Most of the functions are gathered in a table called \lstinline|dplus|. The list of Dplus Lua commands are provided in the next section. Additional explanations and usage examples are provided in the two sections that follow. 

\section{Complete List of Dplus Lua Commands }

The list of D+  \href{http://www.lua.org/}{\texttt{Lua}} commands can be seen in the \texttt{Command Window} of D+ by typing \texttt{dplus.}, and holding the \texttt{Ctrl} and \texttt{Space} keys.

In the following we present the complete list of D+ \href{http://www.lua.org/}{\texttt{Lua}} commands and their arguments or parameters. Optional parameters are marked with [brackets].


\begin{lstlisting}
dplus.openconsole()
\end{lstlisting}
Opens an external console window (for performance and debugging purposes)

\begin{lstlisting}
dplus.closeconsole()
\end{lstlisting}
Closes the external console window

\begin{lstlisting}
dplus.getparametertree()
\end{lstlisting}
Returns the current parameters as a parameter tree object

\begin{lstlisting}
dplus.setparametertree(param_tree)
\end{lstlisting}
Sets the parameter tree as the current parameters

\begin{lstlisting}
dplus.save(filename)
\end{lstlisting}
Saves parameters to .state file

\begin{lstlisting}
dplus.load(filename)
\end{lstlisting}
Loads parameters from .state file

\begin{lstlisting}
dplus.findmodel(name[, container])
\end{lstlisting}
Finds a (geometric) model by name, optionally with an external container (DLL) filename

\begin{lstlisting}
dplus.findamp(name[, container])
\end{lstlisting}
Finds an amplitude model by name, optionally with an external container (DLL) filename

\begin{lstlisting}
dplus.generate([param_tree, save_filename, should_save_amplitudes])
\end{lstlisting}
Generate using the given parameter tree (or nil for current parameters), optionally saving the result to a file. 'should\_save\_amplitudes' determines whether D+ should save the grid amplitudes while generating.

\begin{lstlisting}
dplus.fit(data, param_tree, properties)
\end{lstlisting}
Fit a given parameter tree (or nil for current parameters) to the given data table, using the fitting properties given in 'properties'.

\begin{lstlisting}
dplus.readdata(filename)
\end{lstlisting}
Reads data from a signal file to a Lua table

\begin{lstlisting}
dplus.writedata(filename, data)
\end{lstlisting}
Writes data from a Lua table to a signal file (TSV format)

\begin{lstlisting}
dplus.figure(title, xlabel, ylabel)
\end{lstlisting}
Opens a new figure with a given title, x, and y labels. Returns figure ID for use in dplus.showgraph*

\begin{lstlisting}
dplus.showgraph(data, figure_id, color)
\end{lstlisting}
Plots a signal filename (if data is a string) or table (if data is a Lua table) on a given figure ID, using a specific color (given as a string of the color name). If 'figure\_id' is nil, uses the last opened figure.

\begin{lstlisting}
dplus.{msgbox,mbox,message,messagebox}(message)
\end{lstlisting}
Opens a message box (good for user notifications)

\begin{lstlisting}
dplus.{sleep, wait}(time_ms)
\end{lstlisting}
Wait for the given amount of milliseconds


\section{UI Related Functions}

Some of the functions allow some basic control of the user interface. Of these, some have more complex uses and will be addressed in separate sections.

\paragraph{Opening and closing the console}
\begin{sloppypar}
The console can be opened and closed via the commands \lstinline|dplus.openconsole()| and \lstinline|dplus.closeconsole()| respectively.
\end{sloppypar}

\paragraph{Saving and loading parameter files}
The entire set of model, fitting and camera parameters can be saved to a file\footnote{The saved file is itself a \href{http://www.lua.org/}{\texttt{Lua}} script that contains the parameters in tables.}.
\lstinline|dplus.save(string)| saves the current set of parameters to a file. For example,
\begin{lstlisting}[basicstyle = \small,literate={'}{{'}}1,escapeinside={(*}{*)}]
dplus.save("C:\\path\\to\\file.state")
\end{lstlisting}
will save the current state to \path{C:\path\to\file.state}. Likewise, 
\begin{lstlisting}[basicstyle = \small,literate={'}{{'}}1,escapeinside={(*}{*)}]
dplus.load("C:\\path\\to\\file.state")
\end{lstlisting}
loads the file into the UI and replaces any existing parameters.

\paragraph{Reading scattering curve data files}
Loading a signal to the \hyperref[sec:2DGraph]{\texttt{2D graph}} is achieved by
\begin{lstlisting}
dplus.readdata(filename)
\end{lstlisting}

\paragraph{Message boxes}
A (blocking) dialog box with a specific message can be popped up with any of the following (without \lstinline{dplus.}):
\lstinline{msgbox(string)},
\lstinline{mbox(string)},
\lstinline{message(string)}, or
\lstinline{messagebox(string)}.


\paragraph{Navigating Lua tables} \label{sec:par:printKeys}
Owing to the fact that \href{http://www.lua.org/}{\texttt{Lua}} stores \textit{everything} in a Key-Value table, it is often necessary to find out what the keys are. Using the provided \lstinline{PrintKeys(table)} prints a list of the table keys in the \texttt{Command Window}.

\paragraph{Waiting}
Two identical functions for waiting are provided: \lstinline{sleep(ms)} and \lstinline{wait(ms)}. Both accept an integer and pause the script for that number of milliseconds.


\section{Scripting Related Functions}

When running repetitive tasks (such as generating all the permutations of four parameters at increments of $X$) several types of functions are needed, such as manipulating the parameters, generating models, saving results, etc.

\paragraph{Obtaining the parameters from the GUI}
\begin{sloppypar}
While it is possible to create the parameters from scratch using a script, 
a better method would be to obtain the parameters from the GUI, using  \lstinline{dplus.getparametertree()}. The \hyperref[sec:par:printKeys]{\texttt{PrintKeys}} function can be used to navigate and to obtained tree like so (in the Command Window):
\end{sloppypar}

\begin{lstlisting}[basicstyle = \small,literate={'}{{'}}1,escapeinside={(*}{*)}]
myTree = dplus.getparametertree()
dummyVar = PrintKeys(myTree)
--[[ Prints:
	Size: 4
	Scale
	ScaleMut
	Geometry
	Populations
]]
dummyVar = PrintKeys(myTree.Populations)
--[[ Prints:
	Size: 1
	1
]]
dummyVar = PrintKeys(myTree.Populations[1])
--[[ Prints:
	Size: 3
	PopulationSizeMut
	PopulationSize
	Models
]]
-- etc.
\end{lstlisting}

\paragraph{Setting the parameters to the GUI}
After navigating the tree and changing one of the parameters like
\begin{lstlisting}[basicstyle = \small,literate={'}{{'}}1,escapeinside={(*}{*)}]
myTree.Populations[1].Models[1].Parameters[2][2] = math.pi^math.exp(1)
\end{lstlisting}
the GUI remains ignorant of the change until it's updated.
One way of updating the GUI is to use
\lstinline{dplus.setparametertree(myTree)}.
NOTE: This clears all existing entities/models and creates new ones. If one of the models was a PDB model, then the PDB file will be read again.

\paragraph{Updating the GUI's parameters}
If all that needs to be done is change a few parameters without \textit{adding or removing} parameters, the better option would be to update the GUI parameters.
This way, the aforementioned PDB model does not get reread and any information D+ has created for the model is not lost.
This is done with \lstinline{dplus.updateparametertree(myTree)}.

\paragraph{Generating models}
Modifying parameters is all good, but using them to generate model signals is better.
The command is \lstinline{dplus.generate} and has three different overloads (usage forms).
All forms return a table which contains the ($x,y$) coordinates of the resulting scattering curve graph.
The simple form is \lstinline{res = dplus.generate()}.
This takes the parameters from the GUI and uses them to generate a model of a scattering curve signal.
It is also possible to generate a model without updating the GUI via
\lstinline{res = dplus.generate(myTree, saveFileName)}.
There are some times when one might want to save the amplitude files of the generated model.
This can be achieved by \lstinline{res = dplus.generate(myTree, saveFileName, true)}.

\paragraph{Saving files}
Generating $10^{35}$ models is better, but wouldn't it be amazing if the results were save to a file?
Again, achieved like so (including a bit of error checking):
\begin{lstlisting}[basicstyle = \small,literate={'}{{'}}1,escapeinside={(*}{*)}]
res = dplus.generate()
ddd = dplus.writedata(saveFileName, res)

if not ddd then
	print("There was a problem with file " .. saveFileName .. ".\n");
end
\end{lstlisting}

\paragraph{Fitting}

If saving $10^{35}$ files is amazing, fitting and selecting one file to save would be awesome.
A \href{http://www.lua.org/}{Lua} interface for this purpose  exists and works. The challenge is to choose the right starting fitting parameters. D+ adopted \href{http://ceres-solver.org/}{\texttt{Ceres-Solver}} fitting algorithm using Riddler's method for derivation.

\subsection{Miscellaneous}

In \href{http://www.lua.org/}{Lua} assigning a table to a variable does not copy it.
If we did \lstinline{a = myTree} and then changed a value either in \lstinline{a} or in \lstinline{myTree} the change will be apparent in the other variable.
An actual copy function is provided and used as \lstinline{a = table.copy(myTree}).



\end{document}