
\documentclass[../D+Manual.tex]{subfiles}
\begin{document}

\chapter{Installation}
\label{sec:install}

\begin{quote}
	Computers themselves, and software yet to be developed, will revolutionize the way we learn.\\
	\hspace*{\fill} \textit{Steve Jobs}
\end{quote}

\section{Hardware architecture} \label{sec:hardware}

D+ requires that the CPU is Intel® Advanced Vector Extensions (Intel® AVX) capable. D+ can also use graphics cards (or graphic processing units, GPUs) to accelerate calculations by factors that vary between $\times10$ to $\times1000$.
A GPU is not required, but recommended for performance. 
D+ utilizes NVIDIA GPUs, preferably the higher end products with reasonable (1:3) double- to single- precision floating point ratio (FP64 = $\nicefrac{1}{3}$ FP32) performance.
The cards that are known/assumed to work well are
\begin{enumerate}
	\item GTX 980 (known)
	\item Titan (known)
	\item Titan Black (known)
	\item Titan Z (assumed)
	\item Tesla series (assumed)
	\item Quadro K6000 (assumed)
	\item Quadro K5200 (assumed)
\end{enumerate}

Cards that have low FP64:FP32 performance (1:32 for Maxwell architecture and therefore are not recommended, but will work for most computations) include:
\begin{enumerate}
	\item Titan X
	\item Quadro M6000
\end{enumerate}

\section{Software} \label{sec:software}

\subsection{Windows} \label{sec:windowsInstall}


D+ is freely \href{https://scholars.huji.ac.il/uriraviv/book/d-0}{available} for academic use. You can \href{https://drive.google.com/file/d/1Nu7EhBsu0Hwu3VFXv8IoQmhSRk9u5hJs/view?usp=sharing}{download} the installer of D+ and install the program. 

D+ has a few prerequisites: \href{http://www.microsoft.com/en-us/download/details.aspx?id=17718}{Microsoft .NET 4.0} and \href{http://www.microsoft.com/en-us/download/details.aspx?id=40784}{Visual C++ Redistributable Packages for Visual Studio 2013}.
D+ is a 64 bit application due to memory usage and some library requirements; install the 64 bit versions of the Visual C++ Redistributable Packages.
Also, if a GPU is installed, the proper driver should also be installed to enable CUDA (CUDA 8.0, at the time of writing).
If the D+ installer (an msi file) is used, it will install the Visual C++ Redistributable Packages on its own. On Windows computers with CUDA enabled GPUs, the installer makes sure that the watchdog timer is not enabled. 

Once all the above are installed, the files listed in Table \ref{tbl:installFiles} should be placed in the same directory (the default is \path{C:\Program Files\D+\bin}).
The purposes for including some of these files are indicated Table \ref{tbl:installFiles}.
There are differences in the version that uses a remote server for the computations.
Usage of both versions is practically identical and therefore except for the few and minor differences mentioned in chapter \ref{chp:remoteCalculations}, the remainder of this manual pertains to both.

\begin{table}[h!]
	\centering
	
	\begin{tabularx}{\textwidth}{X X}
	\toprule
	\textbf{Filename} & \textbf{Purpose}\\
	\midrule
	\path{Bin/Aga.Controls.dll}                        & \\
	\path{Bin/ceres.dll}                               & \\
	\path{Bin/cudart64_80.dll}                         & Checks for the graphics card \\
	\path{Bin/curand64_80.dll}                         & \\
	\path{Bin/DebyeCalculator.exe}                     & Calculates the ``ground truth'' for atomic scattering\\
	\path{Bin/DPlus.exe}                               & Launches D+ \\
	\path{Bin/Fit.exe}                                 & \\
	\path{Bin/Generate.exe}                            & \\
	\path{Bin/GetAllMetadata.exe}                      & \\
	\path{Bin/GLView.dll}                              & \\
	\path{Bin/GraphToolkit.dll}                        & \\
	\path{Bin/JSON.lua}                                & \\
	\path{Bin/License.txt}                        & \\
	\path{Bin/lua51.dll}                               & \\
	\path{Bin/lua51-backend.dll}                       & \\
	\path{Bin/LuaInterface.dll}                        & \\
	\path{Bin/Microsoft.WindowsAPICodePack.dll}        & \\
	\path{Bin/Microsoft.WindowsAPICodePack.Shell.dll}  & \\
	\path{Bin/mscorlib.dll}                            & \\	
	\path{Bin/Newtonsoft.Json.dll}                     & \\
	\path{Bin/PDBReaderLib.dll}                        & \\
	\path{Bin/PDBUnits.exe}                            & Helper tool that finds identical copies of segments in a larger PDB file \\
	\path{Bin/Resources/atomicData.txt}               & Used by \path{Bin/PDBUnits.exe} \\
	\path{Bin/SciLexer.dll}                            & \\
	\path{Bin/SciLexer64.dll}                          & \\
	\path{Bin/ScintillaNET.dll}                        & \\
	\path{Bin/Suggest Parameters.exe}                  & A program that suggests parameters for D+ \\
	\path{Bin/WeifenLuo.WinFormsUI.Docking.dll}        & \\
	\path{Bin/xplusbackend.dll}                        & Handles the actual calculations\\
	\path{Bin/xplusfrontend.dll}                       & \\
	\path{Bin/xplusmodels.xrn}                         & \\
	\path{Example Files/Examples}                          & Contains 7 Example subfolders. See Chap. \ref{chp:examples}. \\
	\path{Example Files/Tutorials}                          & Contains 6 Tutorial subfolders. \\
	\path{Example Files/AnomalousInputExample.txt}                          & Anomalous Input File. See Chap. \ref{sec:anomalousScattering}. \\
	\path{Example Files/READ_ME.txt}                          & Explains how to use the examples. \\	
	\path{LuaScripts/}                & Contains 4 examples of  \href{http://www.lua.org/}{Lua} scripts\\
	\path{D+ Manual.pdf}                  & The User's Manual of D+ \\
	\bottomrule
	\end{tabularx}
	\caption{
	The installed files include both files for a local installation, as well as a remote (Windows) installation.
	Depending on the options chosen at installation, you may only see a subset of these files.
	}
	\label{tbl:installFiles}
\end{table}




\subsection{Linux} \label{sec:linuxInstall}

First note that there is no GUI for Linux.
There are precompiled binaries for Ubuntu and CentOS that can be used by the Python API (see Chapter \ref{the-dplus-python-api}).
Download the relevant file (\path{dplus.tar.bz2} or \path{dplus.centos.tar.bz2}) and \path{install.sh}.
Run \path{$ install.sh} alongside the downloaded archive.
Run \path{$ source ~/.bashrc} to set the proper environment variables.


\end{document}