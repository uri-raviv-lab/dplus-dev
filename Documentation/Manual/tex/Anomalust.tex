\documentclass[12pt,a4paper,openany,oneside,oldfontcommands]{memoir}
%\usepackage[utf8]{inputenc}
\usepackage{amsmath}
\usepackage{units}
\usepackage{amsfonts}
\usepackage{amssymb}
\usepackage{gensymb}
\usepackage{graphicx}
\usepackage{memhfixc}
\usepackage{wrapfig}
\usepackage{pdflscape}
\usepackage{afterpage}
\usepackage{algorithm}
\usepackage{algpseudocode}
\usepackage{textcomp}
\usepackage[version=3]{mhchem}

\usepackage{listings}
\usepackage{textcomp}
\usepackage{siunitx}
\usepackage{lipsum}

\usepackage[obeyspaces]{url}

\usepackage{subfiles}
\usepackage{ltablex,tabularx}
\usepackage{longtable,booktabs}


\begin{document}
\subsection{Anomalous scattering in the paper}
\label{sec:AnomScat}
In modern synchrotrons, the energy of the X-ray photons can be varied. If scattering experiments are repeated at multiple wavelengths, and the structure contains atoms whose resonant scattering terms, $f'$ and $f''$, respond to the energy scan, additional structural information can be derived. When modeling atomic models with anomalous scattering the scattering amplitude from an atom $j$ in units of $-r_0$, in vacuum, is given by \cite{als2011elements}:  
\begin{equation}
f^{va}_j\left(q, \lambda\right) = f^{0}_j\left(q\right) + f'_j\left(\lambda\right) + if''_j\left(\lambda\right).
\label{Anomalous}
\end{equation}
Therefore, one needs to provide input in the form of $f'_j\left(\lambda\right)$ and $f''_j\left(\lambda\right)$ for all the atoms whose resonant scattering terms have a significant contribution to the scattering signal in the relevant conditions.
In D+, this is done via a text file loaded when loading an atomic model.
D+ uses the input without any sanity checks, so it is the users responsibility to ensure that proper values are used. In particular, note that in D+, $r_0$ is set to $1$ when $f^{0}_j\left(q\right)$ is computed. 
Input is an $\left\lbrace f'_j, f''_j\right\rbrace$ pair with either a list of the atom indices in the PDB file or an ion type (Eq. \ref{Anomalous} will be applied to all ions of that type).
This type of input may be obtained from X-ray fluorescence spectra and the computer program CHOOCH \cite{evans2001chooch}. See the User's Manual of D+ for more details and usage example.

The anomalous scattering amplitude in solution from atom $j$ is: 
\begin{equation*}
f_j^{sa}\left(q,\lambda\right)=f^{va}_j(q,\lambda)-F^d_j(q)
\end{equation*}
and from a molecule:
\begin{equation*}
F^{as}_{\text{mol}}\left(\vec{q},\lambda\right)=\sum_{i=1}^n f_{i}^{sa}\left(q,\lambda\right)\cdot \exp\left(i\vec{q}\vec{r}_i\right).
\end{equation*}
	
	
	
\section{Anomalous Scattering in the User's Manual} \label{sec:anomalousScattering}
D+ can take anomalous scattering of atoms into account when computing scattering amplitudes of atomic models. Two additional correction terms should be added to get the atomic form factor energy dependence:
\begin{equation*}
	f(q,\lambda) = f^\circ(q) + f'(\lambda) + i f''(\lambda).
\end{equation*}
To enable this option for a given PDB file, choose \texttt{PDB File...} from the dropdown menu in the \texttt{Domain View} window.
Check the \texttt{Anomalous} checkbox \textit{before} pressing the \texttt{Add} button.
An additional \texttt{Open File} dialog will pop up after selecting the PDB file.
Select the text file containing the relevant $f'$ and $f''$ values.
The file should be formated using any combination of the lines below.

%\lstset{style=anomalousFile,}

\begin{lstlisting}
# Energy    2345.82km
# <f'>  <f''>   <atom identifier>
1.2     2.3     Fe2+
#<f'>   <f''>   <atom indices>
1.3     3.4     125     256
#<f'>   <f''>   <name of a group that's defined below>
-3e4    +2.01   $(someRandomGroupName)
#<name of a group>      <list of atom indices>
$(someRandomGroupName)  252 472 7344 313
\end{lstlisting}

Lines that start with '\#' are considered comments and are ignored.
%TODO add an example. What do you mean by atom indices (line in the pdb file?) and group name (CH, CH_2?)? 
The \texttt{Example Files} subfolder of D+ includes more examples. 
\end{document}