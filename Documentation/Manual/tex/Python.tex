
\documentclass[../D+Manual.tex]{subfiles}

\begin{document}
	
\chapter{Python API} \label{chp:Python}

\begin{quote}
	You can start any 'Monty Python' routine and people finish it for you. Everyone knows it like shorthand. \\
	\hspace*{\fill} \textit{Robin Williams}
\end{quote}


\section{The Dplus Python API}\label{the-dplus-python-api}

The D+ Python API allows using the D+ backend from Python, instead of
the ordinary D+ application.

The Python API works on both Windows and Linux.

\subsection{Installation}\label{installation}

Installing the Python API is done using PIP:

\begin{lstlisting}[language=bash,basicstyle=\small, breaklines= true, breakatwhitespace= true]
pip install dplus-api
\end{lstlisting}

The API was tested with Python 3.5 and newer. It \emph{may} work with
older versions of Python, although Python 2 is probably not supported.

\subsubsection{Overview}

Throughout the manual, code examples are given with filenames, like "mystate.state".
To run an example code for yourself, these files must be located in the same directory as the script itself,
or alternately the code can be modified to contain the full path of the file's location.

Throughout the manual, we mention "state files". A state file is a JavaScript Object Notation (JSON) format file (https://www.json.org/), which describes the parameter tree and calculation settings of the D+ computation.
It is unnecessary to write a state file yourself. State files can either be generated from within the python interface (with the function `export\_all\_parameters`),
or created from the D+ GUI (by selecting File$>$Export All Parameters from within the D+ GUI).

The overall flow of the Python API is as follows:
\begin{enumerate}
	\item {The data to be used for the calculation is built by the user in an
		instance of the \texttt{CalculationInput} class (either
		\texttt{GenerateInput} or \texttt{FitInput}). Inputs contain a program \texttt{State}, which includes both program preferences 
		like \texttt{DomainPreferences}, and a parameter tree composed of \texttt{Models}.}
	\item {The calculation input is then passed to a \texttt{CalculationRunner} class
		(either \texttt{LocalRunner} or \texttt{WebRunner}), and the calculation
		function is called (\texttt{generate}, \texttt{generate\_async},
		\texttt{fit}, or \texttt{fit\_async}).}
	\item {The \texttt{CalculationRunner} class returns a 		\texttt{CalculationResult} class.}
\end{enumerate}

\noindent Here is a very simple example of what this might look like in main.py:

\begin{lstlisting}[style=pythonstyle]
from dplus.CalculationInput import GenerateInput
from dplus.CalculationRunner import LocalRunner

calc_data = GenerateInput.load_from_state("mystate.state")
runner = LocalRunner()
result = runner.generate(calc_data)
print(result.graph)
\end{lstlisting}

\noindent A detailed explanation of the class types and their usage follows.

\subsection{CalculationRunner}\label{calculationrunner}

There are two kinds of \texttt{CalculationRunner}s, Local and Web.

The \texttt{LocalRunner} is intended for users who have the D+ executable files
installed on their system. It takes two optional initialization
arguments:

\begin{itemize}
	\tightlist
	\item
	\texttt{exe\_directory} is the folder location of the D+ executables.
	By default, its value is \texttt{None}. On Windows, this value will 
	lead to the python interface searching the registry for an installed D+ on its own, but on linux the executable directory \textit{must} be specified. 
	\item
	\texttt{session\_directory} is the folder where the arguments for the
	calculation are stored, as well as the output results, amplitude
	files, and pdb files, from the C++ executable. By default, its value
	is \texttt{None}, and an automatically generated temporary folder will be used.
\end{itemize}

\begin{lstlisting}[style=pythonstyle]
from dplus.CalculationRunner import LocalRunner

exe_dir = r"C:\Program Files\D+\bin"
sess_dir = r"sessions"
runner = LocalRunner(exe_dir, sess_dir)

#also possible:
#runner = LocalRunner()
#runner = LocalRunner(exe_dir)
#runner = LocalRunner(session_directory=sess_dir)
\end{lstlisting}

The WebRunner is intended for users accessing the D+ server. It takes
two required initialization arguments, with no default values:

\begin{itemize}
	\tightlist
	\item
	\texttt{url} is the address of the server.
	\item
	\texttt{token} is the authentication token granting access to the
	server.
\end{itemize}

\begin{lstlisting}[style=pythonstyle]
from dplus.CalculationRunner import WeblRunner

url = r'http://localhost:8000/'
token = '4bb25edc45acd905775443f44eae'
runner = WebRunner(url, token)
\end{lstlisting}

Both runner classes have the same four methods:

\texttt{generate(calc\_data)}, \texttt{generate\_async(calc\_data)}, \texttt{fit(calc\_data)}, and \texttt{fit\_async(calc\_data)}.

All four methods take the same single argument, \texttt{calc\_data} - an
instance of a \texttt{CalculationData} class.

\texttt{generate} and \texttt{fit} return a \texttt{CalculationResult}.

\texttt{generate\_async} and \texttt{fit\_async} return a \texttt{RunningJob}.

When using \texttt{generate} or \texttt{fit} the program will wait until the call is finished and returned a result, before continuing. Their asynchronous counterparts (\texttt{generate\_async} and \texttt{fit\_async}) allow D+ 
calculations to be run in the background (for example, the user can call \texttt{generate\_async}, tell the program to do other things, and then return and check if the computation is finished). 


\subsection{RunningJob}\label{runningjob}

The user should not be initializing this class. When returned from an
async function (\texttt{generate\_async} or \texttt{fit\_async}) in \texttt{CalculationRunner}, the user can use the following
methods to interact with the \texttt{RunningJob} class:

\begin{itemize}
	\tightlist
	\item
	\texttt{get\_status()}: get a JSON dictionary reporting the job's
	current status
	\item
	\texttt{get\_result(calc\_data)}: get a \texttt{CalculationResult}.
	Requires a copy of the \texttt{CalculationInput} used to create the job. Should
	only be called when the job is completed. It is the user's responsibility
	to verify job completion with \texttt{get\_status()} before calling.
	\item
	\texttt{abort()}: end a currently running job
\end{itemize}

\begin{lstlisting}[style=pythonstyle]
from dplus.CalculationInput import GenerateInput
from dplus.CalculationRunner import LocalRunner


calc_data = GenerateInput.load_from_state("mystate.state")
runner=LocalRunner()
job = runner.generate_async(calc_data)
start_time=datetime.datetime.now()
status = job.get_status()
while status['isRunning']:
	status = job.get_status()
	run_time = datetime.datetime.now() - start_time
	if run_time > datetime.timedelta(seconds=50):
		job.abort()
		raise TimeoutError("Job took too long")
result = job.get_result(calc_data)
\end{lstlisting}

\subsection{CalculationInput}\label{calculationinput}

There are two kinds of \texttt{CalculationInput}, \texttt{FitInput}, and \texttt{GenerateInput}.

\texttt{GenerateInput} contains an instance of a \texttt{State} class and an \texttt{x} (or a \texttt{q})
vector. It is used to generate the intensity signal, \texttt{I}, of a given parameter tree (within the \texttt{State}).

\texttt{FitInput} contains a \texttt{State} class, an \texttt{x} (or $q$) vector, and a \texttt{y} (or \texttt{I}) vector, representing a signal to be fitted.
It is used to fit a parameter tree (within the \texttt{State} to the signal (\texttt{I(q)}).

The \texttt{State} class is described in the next section.

The \texttt{x} (or \texttt{q}) and \texttt{y} (or \texttt{I}) vectors are simply lists of floating point coordinates. They
can be generated from parameters in the \texttt{State} class or loaded from a
file.
%{[}This section will be expanded on in future, when additional options for these vectors will be added{]}

\texttt{CalculationInput} has the following methods:

\begin{itemize}
	\tightlist
	\item
	\texttt{get\_model}: get a model by either its \texttt{name} or its pointer,
	\texttt{model\_ptr}.
	\item
	\texttt{get\_models\_by\_type}: returns a list of \texttt{Models} with
	a given \texttt{type\_name}, for example, \texttt{UniformHollowCylinder}.
	\item
	\texttt{get\_mutable\_params}: returns a list of \texttt{Parameters}
	in the \texttt{State} class, whose property \texttt{mutable} is \texttt{True}.
	\item 
	\texttt{get\_mutable\_parameter\_values}: returns a list of floats, matching the values of the \texttt{mutable parameters}.
	\item
	\texttt{set\_mutable\_parameter\_values}: given a list of floats, sets the \texttt{mutable parameters} of the \texttt{State} (in the order given by 
	\texttt{get\_mutable\_parameter\_values}).
	\item
	\texttt{export\_all\_parameters}: given a filename, will save the calculation \texttt{State} to that file.
\end{itemize}


In addition, all \texttt{CalculationInputs} have the property \texttt{use\_gpu}, which can be set to \texttt{True} or \texttt{False} (running fitting with
\texttt{use\_gpu} set to \texttt{False} is not recommended).

A new instance of \texttt{GenerateInput} can be created simply by calling its constructor with a \texttt{State}:

\begin{lstlisting}[style=pythonstyle]
from dplus.CalculationInput import GenerateInput
s=State()
gen_input=GenerateInput(s)
\end{lstlisting}


%TODO add examples of args, json args
In addition, \texttt{GenerateInput} has the following static methods to create an instance of
\texttt{GenerateInput}:

\begin{itemize}
	\tightlist
	\item
	\texttt{load\_from\_state\_file(filename)} receives the location (or path) of a file that contains a serialized parameter tree (\texttt{State}).
	\item
	\texttt{load\_from\_PDB} receives the location (or path) of a PDB file, and
	automatically creates a guess at the best \texttt{State} parameters based on the PDB file.
\end{itemize}


\begin{lstlisting}[style=pythonstyle]
from dplus.CalculationInput import GenerateInput
gen_input=GenerateInput.load_from_state_file('sphere.state')
\end{lstlisting}

A new instance of \texttt{StateInput} can be created by calling its constructor, and either:

1. \texttt{x}, \texttt{y}: two arrays, an \texttt{x} array and a \texttt{y} array, or

2. \texttt{graph}: a single dictionary, with \texttt{x} values as keys for the \texttt{y} values.


\begin{lstlisting}[style=pythonstyle]
from dplus.CalculationInput import FitInput, load_x_and_y_from_file

x,y=load_x_and_y_from_file("signal_file.out")
state=State()
fit_input=FitInput(state, x=x, y=y)
\end{lstlisting}

\texttt{FitInput} also has the following static method to create an instance of \texttt{FitInput} 

\begin{itemize}
	\item \texttt{load\_from\_state\_file(filename)} receives the location (or path) of a file that contains a serialized parameter tree (state)
\end{itemize}

\begin{lstlisting}[style=pythonstyle]
from dplus.CalculationInput import FitInput
fit_input=FitInput.load_from_state_file('sphere.state')
\end{lstlisting}

\subsubsection{State}\label{state}

The \texttt{state} class contains an instance of each of three classes:
\texttt{DomainPreferences}, \texttt{FittingPreferences}, and \texttt{Domain}. They are described in
the upcoming sections.

It has the methods \texttt{get\_model}, 
\texttt{get\_models\_by\_type},
and \texttt{get\_mutable\_params},
 
\texttt{get\_mutable\_parameter\_values},
\texttt{set\_mutable\_parameter\_values}, and \texttt{export\_all\_parameters}, just as \texttt{CalculationInput} does. In fact, \texttt{CalculationInput} simply
invokes these functions from \texttt{State} when they are called from
\texttt{CalculationInput}.

\texttt{State}, \emph{and every class and sub class contained within \texttt{State}} (for example, 
\texttt{preferences}, \texttt{models}, \texttt{parameters}), all have the functions
\texttt{load\_from\_dictionary} and \texttt{serialize}.

\texttt{load\_from\_dictionary} sets the values of the various fields within a class
to match those contained within a suitable dictionary. It can behave
recursively as necessary, for example, with a model that has children.

\texttt{serialize} saves the contents of a class to a dictionary. Note
that there may be additional fields in the dictionary beyond those
described in this document, because some defunct (outdated, irrelevant,
or not-yet-implemented) fields are still saved in the serialized
dictionary.

\texttt{add\_model} is a convenience function to help add models to the parameter tree of a \texttt{State}. It receives the model and optionally 
a population index (default 0), and will insert that model into the population.

\texttt{add\_amplitude} is a convenience function specifically for adding instances of the \texttt{Amplitude} class, described below. 
It creates an instance of an \texttt{AMP} file with the filename of the \texttt{Amplitude}. Then, in addition to calling \texttt{add\_model} with that \texttt{AMP} file, it also changes the \texttt{DomainPreferences} of the \texttt{State} (specifically, \texttt{grid\_size}, \texttt{q\_max}, and \texttt{use\_grid}), to match the properties of \texttt{Amplitude}.
It returns the \texttt{AMP} file it created.


\paragraph{DomainPreferences}\label{domainpreferences}

The \texttt{DomainPreferences} class contains properties that are copied from the
D+ interface. Their usage is explained in the D+ documentation.

We create a new instance of \texttt{DomainPreferences} by calling the python
initialization function:

\texttt{dom\_pref=\ DomainPreferences()}

There are no arguments given to the initialization function, and all the
properties are set to default values:

\begin{longtable}[]{@{}l>{\raggedright}p{3.75cm}>{\raggedright}p{8cm}@{\extracolsep{\fill}}}
	\toprule
		Property Name & Default Value & Allowed values \tabularnewline
	\midrule
	\endhead
	\texttt{signal\_file} & ``'' & ``'', or a valid file location \tabularnewline
	\texttt{Convergence} & 0.001 & \tabularnewline
	\texttt{grid\_size} & 100 & Even integer greater than 20 \tabularnewline
	\texttt{orientation\_iterations} & 100 & \tabularnewline
	orientation\_method & \texttt{"Monte Carlo (Mersenne Twister)"} & \texttt{"Monte Carlo (Mersenne Twister)"}, \texttt{"Adaptive (VEGAS) Monte Carlo"}, \texttt{"Adaptive Gauss Kronrod"} \tabularnewline
	\texttt{use\_grid} & \texttt{"False"} & \texttt{"True"}, \texttt{"False"} \tabularnewline
	\texttt{q\_max} & 7.5 & Positive number. If a signal file is provided, must match the highest \texttt{x} (or \texttt{q}) value\tabularnewline
	\bottomrule
\end{longtable}

Any property can then be easily changed, for example:

\texttt{dom\_pref.q\_max=\ 10}

If the user tries to set a property to an invalid value (for example,
setting $q_{\text{max}}$ to something other than a positive number) they will get an error.

If a signal file is provided, the value of $q_{\text{max}}$ will automatically be
set to the highest x (or q) value in the signal file.

\paragraph{Fitting Preferences}\label{fitting-preferences}

The \texttt{FittingPreferences} class contains properties that are copied from
the D+ interface. Their usage is explained in the D+ documentation.

We create a new instance of \texttt{FittingPreferences} by calling the python
initialization function:

\texttt{fit\_pref=\ FittingPreferences()}

There are no arguments given to the initialization function, and all the
properties are set to default values:


\afterpage{%
	\clearpage% Flush earlier floats (otherwise order might not be correct)
%	\thispagestyle{empty}% empty page style (?)
	\begin{landscape}% Landscape page
		\centering % Center table
\begin{longtable}[l]{
		@{\extracolsep{\fill}}p{3.5cm}
		@{\extracolsep{\fill}} c% p{2.5cm}
		@{\extracolsep{\fill}} p{6.3cm}
		@{\extracolsep{\fill}} p{6.5cm} <{\raggedright}
		@{\extracolsep{\fill}}}
	\toprule
	
	Property Name	& 	Default Value	&  Allowed Values	& 	Required when	\tabularnewline
	\midrule
	\endhead
	
	\texttt{convergence}	& 	0.1	& 	Positive numbers	& 	\tabularnewline
	
	\texttt{der\_eps}	& 	0.1	& 	Positive numbers	& 	\tabularnewline
	
	\texttt{fitting\_iterations}	& 	20	& 	Positive integers & \tabularnewline
	
	\texttt{step\_size}	& 	0.01	& 	Positive numbers & 	\tabularnewline
	
	\texttt{loss\_function}	& 	\texttt{"Trivial Loss"}	& 	\texttt{"Trivial Loss"}, \texttt{"Huber Loss"}, \texttt{"Soft L One Loss"}, \texttt{"Cauchy
	Loss"}, \texttt{"Arctan Loss"}, \texttt{"Tolerant Loss"}	& \tabularnewline
	
	\texttt{loss\_func\_param\_one}	& 	0.5	& 	Number	& 	Required for any types of \texttt{loss\_function}, except \texttt{"Trivial Loss"}	\tabularnewline
	
	\texttt{loss\_func\_param\_two}	& 	0.5	& 	Number	& 	Required when the \texttt{loss\_function} is \texttt{"Tolerant Loss"} \tabularnewline
	
	\texttt{x\_ray\_residuals\_type}	& 	\texttt{"Normal Residuals"}	& 	\texttt{"Normal Residuals"}, \texttt{"Ratio Residualsv}, \texttt{"Log Residuals"}	& \tabularnewline
	
	\texttt{minimizer\_type}	& \texttt{"Trust Region"}	& \texttt{"Line Search"}, \texttt{"Trust Region"}	& \tabularnewline
	
	\texttt{trust\_region\_strategy\_type} & \texttt{"Dogleg"} & \texttt{"Levenberg-Marquardt"}, \texttt{"Dogleg"} & \texttt{minimizer\_type} is \texttt{"Trust Region"}	\tabularnewline
	
	\texttt{dogleg\_type} &  \texttt{"Traditional Dogleg"} & \texttt{"Traditional Dogleg"}, \texttt{"Subspace Dogleg"} & 	\texttt{trust\_region\_strategy\_type} is \texttt{"Dogleg"} \tabularnewline
	
	\texttt{line\_search\_type} & \texttt{"Armijo"}	& \texttt{"Armijo"}, \texttt{"Wolfe"} & \texttt{minimizer\_type} is \texttt{"Line Search"} \tabularnewline
	
	\texttt{line\_search\_direction\_type} & \texttt{"Steepest Descent"} & \texttt{"Steepest Descent"}, \texttt{"Nonlinear Conjugate
	Gradient"}, \texttt{"L-BFGS"}, \texttt{"BFGS"} & \texttt{minimizer\_type} is \texttt{"Line Search"}. If \texttt{line\_search\_type} is \texttt{"Armijo"},
	cannot be \texttt{"BFGS"} or \texttt{"L-BFGS"}. \tabularnewline
	
	\texttt{nonlinear\_conjugate\_gradient\_type} & 	``'' & 	\texttt{"Fletcher Reevesv"}, \texttt{"Polak Ribirere"},\texttt{"Hestenes Stiefel"} & \texttt{linear\_search\_direction\_type} is \texttt{"Nonlinear Conjugate Gradient"}	\tabularnewline
	\bottomrule
\end{longtable}
\label{Fittable}
	\end{landscape}
	\clearpage% Flush page
}
\newpage


Any property can then be easily changed, for example,

\texttt{fit\_pref.convergence=\ 0.5}

If users try to set a property to an invalid value they will get an
error.

\paragraph{Domain}\label{domain}

The Domain class describes the parameter tree.

The root of the tree is the \texttt{Domain} class. This class contains an
array of \texttt{Population} classes. Each \texttt{Population} can
contain a number of \texttt{Model} classes. Some models have children, which are also models.

\subparagraph{Models}\label{models}

\texttt{Domain} and \texttt{Population} are two special kinds of models.

The \texttt{Domain} model is the root of the parameter tree, which can contain
multiple populations. Populations can contain standard types of models.

The available standard model classes are:

\begin{itemize}
	\tightlist
	\item
	\texttt{UniformHollowCylinder}
	\item
	\texttt{Sphere}
	\item
	\texttt{SymmetricLayeredSlabs}
	\item
	\texttt{AsymmetricLayeredSlabs}
	\item
	\texttt{Helix}
	\item
	\texttt{DiscreteHelix}
	\item
	\texttt{SpacefillingSymmetry}
	\item
	\texttt{ManualSymmetry}
	\item
	\texttt{PDB} - a PDB file
	\item
	\texttt{AMP} - an amplitude grid file
\end{itemize}

You can create any model by calling its initialization.

Please note that models are dynamically loaded from those available in D+. 
Therefore, your code editor may underline the model in red even if the model exists.

Models have \texttt{Location Parameters} and \texttt{Extra Parameters}. Some models (that support layers) also contain \texttt{Layer Parameters}.
These are all collection of instances of the \texttt{Parameter} class, and can be accessed from 
\texttt{model.location\_params}, \texttt{model.extra\_params}, and \texttt{model.layer\_params}, respectively.

All of these can be modified. They are accessed using dictionaries

Example:

\begin{lstlisting}[style=pythonstyle]
from dplus.DataModels.models import UniformHollowCylinder

uhc=UniformHollowCylinder()
uhc.layer_params[1]["Radius"].value=2.0
uhc.extra_params["Height"].value=3.0
uhc.location_params["x"].value=2
\end{lstlisting}


For additional information about which models have layers and what the various parameters available for each model are, please consult the D+ User's Manual.

\textbf{Parameters}

The \texttt{Parameter} class contains the following properties:

value: a float whose default value is \texttt{0}

sigma: a float whose default value is \texttt{0}

mutable: a boolean whose default value is \texttt{False}

constraints: an instance of the \texttt{Constraints} class, by default it is the
default \texttt{Constraints}  (see below).

Usage:

p=Parameter()  - creates a parameter with value: 0, sigma: 0, mutable: \texttt{False}, and the default constraints.

p=Parameter(7) - creates a parameter with value: 7, sigma: 0, mutable: \texttt{False}, and the default constraints.

p=Parameter(sigma=2)   - creates a parameter with value: 0, sigma: 2, mutable: \texttt{False}, and the default constraints.

p.value= 4  - modifies the value to be 4.

p.mutable=True  - modified the value of mutable to be \texttt{True}.

p.sigma=3 - modifies sigma to be 3.

p.constraints=Constraints(min\_val=5) sets a \texttt{Constraints} instance whose minimum value (min\_val) is 5.

\textbf{Constraints}

The \texttt{Constraints} class contains the following properties:

MaxValue: a float whose default value is \texttt{infinity}

MinValue: a float whose default value is \texttt{-infinity}

The usage is similar to \texttt{Parameter} class, for example:

c=Constraints(min\_val=5) - sets a \texttt{Constraint} instance whose minimum value is 5 and its maximum value is the default (\texttt{infinity}).


\subsection{CalculationResult}\label{calculationresult}

The CalculationResult class is returned by the CalculationRunner. The
user should generally not be instantiating the class themselves.

The class has the following properties accessible:

\begin{itemize}
	\tightlist
	\item
	`graph': an OrderedDict whose keys are x (or q) values and whose values are y (or I)
	values.
	%TODO check if there is an option to have a list of x as well
	\item
	`y': The raw list of y (or I) values from the results JSON
	\item
	`headers': an OrderDict of headers, whose keys are ModelPtrs and whose
	values are the header associated. This property is not necessarily
	present in fitting results
	\item
	`parameter\_tree': A JSON of parameters (can be used to create a new
	state with state's load\_from\_json). Only present in fitting, not
	generate, results
	\item
	`error' : returns the JSON error report from the dplus run
\end{itemize}

In addition, Calculation results has the following public functions:

\begin{itemize}
	\tightlist
	\item
	`get\_amp(model\_ptr, destination\_folder)': returns the file location
	of the amplitude file for a given model\_ptr. destination\_folder has a
	default value of None, but if provided, the amplitude file will be
	copied to that location, and then have its address returned
	\item
	`get\_pdb(mod\_ptr, destination\_folder)': returns the file location
	of the pdb file for a given model\_ptr. destination\_folder has a
	default value of None, but if provided, the pdb file will be copied to
	that location, and then have its address returned
\end{itemize}

\subsubsection{Amplitude}\label{amplitude}

Within the CalculationResult module there is an Amplitude class. It has
a static method, `load', which receives a filename and $q_{\text{max}}$ value and
creates an instance of the Amplitude class.

Alternately one can create an empty instance of Amplitude and then call
the function `read\_amp', which accomplishes the same thing.

The Amplitude class contains two properties, `arr'
that stores the Amplitude values, and
`headers', which stores a list of headers.

The function `save' saved the contents of 'arr' and headers to a new .amp
file.

\begin{lstlisting}[style=pythonstyle]
my_amp=Amplitude.load('myamp.amp', 5)
#insert various modifications of amplitude here
my_amp.save('myamp-modified.amp')
\end{lstlisting}

\subsection{Additional Usage examples}\label{additional-usage-examples}

From the Dplus GUI it is possible to create a state file by selecting
File\textgreater{}Export All Parameters. Alternately one can create a
State by hand, by adding populations, models, fittingpreferences, etc.
In addition, there is an option for generating a pdb, load\_from\_pdb.
It requires the address of the pdb file, and the value of $q_{\text{max}}$. It automatically populates the rest of the state with
reasonable default values.

\textbf{\emph{Example One}}

\begin{lstlisting}[style=pythonstyle]
from dplus.CalculationInput import FitInput
from dplus.CalculationRunner import LocalRunner

exe_directory = r"..\x64\release"
sess_directory = r"session"
runner= LocalRunner(exe_directory, sess_directory)


input=FitInput.load_from_state('spherefit.state')
result=runner.fit(input)
print(result.graph)
\end{lstlisting}

\textbf{\emph{Example Two}}

\begin{lstlisting}[style=pythonstyle]
from dplus.CalculationInput import GenerateInput
from dplus.CalculationRunner import LocalRunner
from dplus.DataModels import ModelFactory, Population
from dplus.State import State
from models import UniformHollowCylinder

sess_directory = r"session"
runner= LocalRunner(session_directory=sess_directory)

uhc=UniformHollowCylinder()
s=State()
s.Domain.populations[0].add_model(uhc)

caldata = GenerateInput(s)
result=runner.generate(caldata)
print(result.graph)
\end{lstlisting}

\textbf{\emph{Example Three}}

\begin{lstlisting}[style=pythonstyle]
runner=LocalRunner()

caldata=GenerateInput.load_from_PDB('1JFF.pdb', 5)
result=API.generate(caldata)
print(result.graph)
\end{lstlisting}

\textbf{\emph{Example Four}}

\begin{lstlisting}[style=pythonstyle]
API = test_local()
input = GenerateInput.load_from_state('../example files/uhc.state')
cylinder = input.get_model("Cylinder")

print("Original radius is ", cylinder.layer_params[1]['Radius'].value)
result = API.generate(input)

fit_input = FitInput(input.state, result.graph)
cylinder = fit_input.get_model("Cylinder")
cylinder.layer_params[1]['Radius'].value = 2
cylinder.layer_params[1]['Radius'].mutable = True

fit_result = API.fit(fit_input)
print(fit_result.parameter_tree)
fit_input.combine_results(fit_result)
print("Result radius is ", cylinder.layer_params[1]['Radius'].value)
\end{lstlisting}

\subsubsection{Python Fitting}\label{python-fitting}

It is possible to fit a curve using the results from Generate and
numpy's built in minimzation/curve fitting functions. This is a new
functionality that is sill very much under development. An example
follows:

\begin{lstlisting}[style=pythonstyle]
import os
import json
from dplus.CalculationInput import GenerateInput, FitInput
from dplus.CalculationData import CalculationData
from dplus.API import LocalCalculationAPI
from dplus.CalculationResult import CalculationResult
from dplus.Fit import Fitter
from dplus.DataModels import State

statePath = r"/home/avi1604/Downloads/FitTest/state.state"
state = State()
state.load_from_json(json.load(statePath))
filename = state.DomainPreferences['SignalFile']
x, y = FitInput._load_x_and_y_from_file(filename)
fitInput = FitInput(state, x=x, y=y)

exe_file = r"/home/avi1604/Downloads/bin"
api = LocalCalculationAPI(exe_file)

fitter = Fitter(api)
result = fitter.run(fitInput)
\end{lstlisting}

\end{document}