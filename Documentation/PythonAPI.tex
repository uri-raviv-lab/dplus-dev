





\documentclass[12pt]{article}

\usepackage{listings}
\lstset{language=c++}
\lstMakeShortInline[language=C++,basicstyle=\ttfamily]~

\usepackage{float}
\floatstyle{boxed}
\restylefloat{figure}

\usepackage{hyperref}

\usepackage{todonotes}

\usepackage{amssymb}

\usepackage{longtable}

\newcommand{\avinote} [2][]{\todo[#1, color=blue!30]{\textbf{Avi}: #2}}
\newcommand{\urinote} [2][]{\todo[#1, color=yellow!30]{\textbf{Uri}: #2}}
\newcommand{\itaynote}[2][]{\todo[#1, color=green!30]{\textbf{Itay}: #2}}
\newcommand{\talnote} [2][]{\todo[#1, color=red!30]{\textbf{Tal}: #2}}
\newcommand{\devnote} [2][]{\todo[#1, color=purple!30]{\textbf{Dev}: #2}}

%opening
\title{The Python API}
\author{Devora Witty}

\begin{document}
	
	\maketitle
	
	\begin{abstract}
		
		In place of the DPlus graphical interface, it is possible to interact with the DPlus backend/engine by way of a python interface (intended for scripting etc). This document describes the Python interface.
		
	\end{abstract}
	
	\section{Intoduction to the Python Interface}
	The DPlus Python Wrapper works as follows:
	1. An API object is created (using arguments specific to the type of API)
	2. The data for the calculation is used to build a CalculationData object
	3. The API function 'generate' is called, passing the CalculationData object as an argument
	3. the API returns a CalculationResult object, upon which further manipulations
	are then possible
	
	
	
	
	\section{Installation}
	Blablalbla
	
	\section{Classes}
	\subsection{CalculationData}
\subsubsection{creating a CalculationData object}
CalculationData has two static methods (currently) for
creating a CalculationData object from a file.

1. ~load_from_state~ receives the address of a state file. It is possible to create
state files in Dplus's graphical UI simply by selecting 'export all parameters'
in the file menu.
2. ~load_from_pdb~ receives the address of a pdb file. For now, it then plugs this pdb file
into an existing, default state tree.

CalculationData builds an x vector on the basis of the state arguments it is provided.
It then passes along a dictionary bundling the state and the x vector when its property 'args' is called.

\subsubsection{using CalculationData object}

the created CalculationData object should be passed as-is to the API object in the call to generate
	
	\subsection{API}
\subsubsection{Creating an API object}

There are currently two kinds of APIs available:

1. LocalAPI- this can be run from a computer which has the generate executable. Its constructor requires as
its first argument the full path of the directory in which the generate executable is located. A second, optional
argument is the full path of where results from the calculation will be stored-- by default, they will be stored
in the computer's temporary folder

2. WebAPI- this can be run from any computer with an internet connection.
Its constructor takes two required arguments: The address of the server, and an authorization key
for accessing the server

\subsubsection{using API object}
once the API object has been created, it can be used by calling 'generate' with a CalculationData object
	
	\subsection{CalculationResult}
\subsubsection{Creating a CalculationResult object}
The user is not responsible for creating CalculationResult, it is created by the API
\subsubsection{Using CalculationResult object}
Currently, it is only possible to use the CalculationResult object to view/print its graph
and headers properties. More functionality will be added in the future.
	
	\section{Usage Examples}
\subsection{sample code: local API with state file}
~from dplus.CalculationData import CalculationData~\\
~from dplus.API import LocalCalculationAPI~\\
~data=CalculationData.load_from_state(r'statefile.state')~\\
~API=LocalCalculationAPI(r"address/of/executable")~\\
~result = API.generate(data)~\\
~print(result.graph)~\\

\subsection{sample code: web API with pdb file}


~from dplus.CalculationData import CalculationData~\\
~from dplus.API import LocalCalculationAPI~\\
~data=CalculationData.load_from_pdb(r'pdbfile.pdb')~\\
~API=WebCalculationAPI(r"http://some.web.address", "T0K3N")~\\
~result = API.generate(data)~\\
~print(result.graph)~\\
	
	
\end{document}


